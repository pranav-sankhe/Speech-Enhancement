\chapter{Introduction}

\section{Task Description}

The task of speech enhancement by suppresion of acoustic noise has been explored here. The problem of speech enhancement has been heavily researched on owing to its immense importance in plethora of audio processing tasks like speech recognition and compression, restoration of analog audio recordings, etc. In particular, the following work focuses on enhancement of speech recordings and the tranfer of the techniques from speech to music domain sure exists but is not trivial. This transfer has not been explictly discussed in this paper but analysis of audio supression in music signals is an important and interesting research in it's own right.  \\ 

If we do not assume any particular nature whatsoever of the noise we aim to supress in a speech signal, the task of denoising becomes an extremely challenging. In this paper, the noise has been assumed to be additive and stationary i.e. white noise (broadband noise, like tape hiss), colored noise, and different kinds of narrowband noises. \\ 

\section{Methods Investigated}
Specificaly, two methods of noise suppresion have been explored here namely, Spectral Subtraction and Wiener Filtering. The implementation of both of these methods and their pecularities have been discussed in the subsequent sections. It is observed that the techniques considered are not equally good at removing noises of different nature. The choice of the technique significantly impacts the performance given a specific application and a specfic noise structure. The paper concludes with the advantages and the disadvantages of both methods and suggest some plausible improvisitions that can potentially improve the noise supression.





